\documentclass[10pt,letterpaper,twocolumn]{article}
\usepackage[utf8]{inputenc}
\usepackage{amsmath}
\usepackage{amsfonts}
\usepackage{amssymb}
\usepackage{mathtools}
\usepackage{physics}
\usepackage{fancyhdr}
\author{Michael Bradley}
\title{Power Regulators}

\pagestyle{fancy}
\fancypagestyle{plain}{}
\fancyhf{}
\lhead{Little Papers Pile}
\rhead{\thepage}

\begin{document}
	\maketitle
	\begin{abstract}
		Often times machines need to have their power provided in a non-jarring way that prevents damage to the machine.
		This is where a simple algorithm comes in the adds and order of smoothness to an incoming power supply.
		The document here describes a simple method for achieving this.
	\end{abstract}
	
	\section*{Single Phase Regulator}
	A single phase regulator makes a input power stream $p_0$ into a continuous output power stream $p_1$.
	A regulator has an internal energy storage $\phi$.
	The output power $p_1$ is equal in magnitude to the current internal energy $\phi$.
	This results in a continuous curve that moves between changes in $p_0$.
	A simple differential equation can describe this effect.
	\[\dv{\phi}{t} = p_0 - p_1\]
	And since $\phi = p_1$,
	\[\dv{p_1}{t} = p_0 - p_1\]
	
	Using a little bit of calculus, we can solve for $p_1$ in terms of a constant input power $p_0$, starting internal energy $\phi_0$, and time from start $t$.
	\[p_1 = p_0 \pm e^{\ln(\pm\qty(\phi_0 - p_0)) - t}\]
	The ``plus or minus'' symbols are there for when
	\[\phi_0 - p_0\]
	is greater than or less than zero in order to prevent plugging in negative numbers into the natural logarithm.
	
	\section*{Two Phase Regulator}
	A two phase regulator will not only ensure that the output power is continuous, but it will also guarantee that it is differentiable, or smooth.
	To construct it is as simple as connecting 2 single phase regulator to one another.
	The following differential equations are produced.
	\begin{align*}
		\dv{p_1}{t} &= p_0 - p_1 \\
		\dv{p_2}{t} &= p_1 - p_2
	\end{align*}
	
	Solving this system for constant input power, starting internal energy, and time passed is as simple as plugging the last solution for the single phase regulator into itself.
	
	\section*{N-Phase Regulator}
	This leads to the next step, which is a regulator with $n$ many single phase regulators.
	The system of differential equations is just as expected.
	\begin{align*}
		\dv{p_1}{t} &= p_0 - p_1 \\
		&\vdotswithin{=} \\
		\dv{p_n}{t} &= p_{n-1} - p_n
	\end{align*}
	
	\section*{Regulator Reaction Speed}
	The reaction speed of a regulator, or how fast the output power changes when the input power changes, can be modified during any phase by multiplying the differential relationship by a constant.
	A sequence of constants is chosen,
	\[\qty(k_1, k_2, \dots, k_n)\]
	And the differential equations are modified as shown.
	\begin{align*}
		\dv{p_1}{t} &= k_1 \qty(p_0 - p_1) \\
		&\vdotswithin{=} \\
		\dv{p_n}{t} &= k_n \qty(p_{n-1} - p_n)
	\end{align*}
	
	Larger and smaller $k$ values lead to a faster and slower reacting regulator respectively.
	
	\section*{Extra Notes}
\end{document}